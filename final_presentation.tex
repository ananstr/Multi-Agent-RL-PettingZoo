\documentclass[aspectratio=169]{beamer}
\usepackage[style=apa, backend=biber]{biblatex}
\addbibresource{latex/references.bib}
\AtBeginBibliography{\scriptsize}
\usepackage{svg}
\usepackage{booktabs}
\usepackage{graphicx}
\usepackage{adjustbox}
\usepackage{animate} % Animate GIFs
\usetheme{metropolis}

\title{Resource Competition in Multi-Agent Reinforcement Learning (MARL)}
\subtitle{A Literature Review \& Implementation}
\date{\today}
\author{Anastasia Chernavskaia, Moritz Peist, Nicolas Rauth}
\institute{\includesvg[width=4cm]{latex/imgs/BSE Barcelona Graduate School of Economics.svg}}

\begin{document}

% Title slide
\begin{frame}{}
    \titlepage
\end{frame}

% Logo on all slides
\setbeamertemplate{frametitle}
{
    \nointerlineskip
    \begin{beamercolorbox}[sep=0.3cm,wd=\paperwidth]{frametitle}
        \strut\insertframetitle\strut
        \hfill
        \raisebox{-0.8mm}{\includesvg[width=1cm]{latex/imgs/BSE Barcelona Graduate School of Economics.svg}}
    \end{beamercolorbox}
}

% Outline
\begin{frame}
    \frametitle{Outline}
    \tableofcontents
\end{frame}

% ========== SECTION 1: INTRODUCTION ==========
\section{Introduction: Why Multi-Agent RL Matters}

\begin{frame}{The Multi-Agent Challenge}
    
    \begin{columns}
        \begin{column}{0.45\textwidth}
            \textbf{Traditional RL}: One agent learns in a static environment \\
            \textbf{Multi-Agent RL}: Multiple agents learning simultaneously \\
            \textbf{Key Challenges:}
            \begin{itemize}
                \item Non-stationarity (environment changes as agents learn)
                \item Credit assignment (who contributed to the outcome?)
                \item Competition vs cooperation
                \item Scalability with increasing agents
            \end{itemize}
        \end{column}
        \begin{column}{0.55\textwidth}
            \centering
            %\includegraphics[width=\textwidth]{training_dashboard_with_coord_ma.gif}
            \animategraphics[controls, 
            autoplay, % Starts playing immediately
            autoresume, % Resumes after page focus returns
            autopause, % Pauses when page loses focus
            width=\linewidth]{4}{latex/imgs/dashboard/frame_}{0}{49}
            \small{Multi-agent coordination in action}
        \end{column}
    \end{columns}
\end{frame}

\begin{frame}{Research Focus \& Approach}
    \textbf{Project Approach: Literature Review} + \textbf{Code Replication} \\
    Focus on resource competition and cooperation dynamics
    \vfill
    \begin{columns}
        \begin{column}{0.5\textwidth}
            \textbf{Literature Survey:}
            \begin{itemize}
                \item MADDPG framework
                \item Value decomposition methods  
                \item Sequential social dilemmas
                \item Industry applications
            \end{itemize}
        \end{column}
        \begin{column}{0.5\textwidth}
            \textbf{Implementation:}
            \begin{itemize}
                \item Simple Spread environment
                \item PPO algorithm
                \item 3 agents coordinating
                \item Resource allocation challenge
            \end{itemize}
        \end{column}
    \end{columns}
    \vfill
    \textbf{Key Reference:} OpenAI's "Learning to cooperate, compete, and communicate" \autocite{lowe_learning_2017}
\end{frame}

% ========== SECTION 2: FOUNDATIONAL FRAMEWORKS ==========
\section{Foundational Frameworks}

\begin{frame}{The MADDPG Revolution}
    
    \begin{block}{Core Innovation: Centralized Training, Decentralized Execution}
        Train with global information, execute with local observations only
    \end{block}
    \vfill
    \begin{columns}
        \begin{column}{0.6\textwidth}
            \textbf{MADDPG Architecture:}
            \begin{itemize}
                \item Each agent: own actor $\mu_i(o_i|\theta_i)$
                \item Centralized critic: $Q_i^\mu(x, a_1, \ldots, a_N)$
                \item Access to all agents' actions during training
                \item Solves the non-stationarity problem
            \end{itemize}
            \vfill
            \textbf{Key Insight:} Environment becomes stationary from critic's perspective \autocite{lowe_multi-agent_2017}
        \end{column}
        \begin{column}{0.4\textwidth}
            \centering
            %\includegraphics[width=\textwidth]{openai_maddpg_diagram.png}
            \animategraphics[controls, 
            autoplay, % Starts playing immediately
            autoresume, % Resumes after page focus returns
            autopause, % Pauses when page loses focus
            width=\linewidth]{20}{latex/imgs/oai-diagram/nipsdiagram_2-}{0}{168}
            \small{MADDPG training architecture}
        \end{column}
    \end{columns}
\end{frame}

\begin{frame}{Value Decomposition Methods}
    
    \textbf{Problem:} How do we assign credit in team rewards?
    
    \vfill
    
    \begin{columns}
        \begin{column}{0.5\textwidth}
            \textbf{QMIX Approach:} \autocite{rashid_monotonic_2020}
            \begin{itemize}
                \item Individual Q-values: $Q_i(o_i, a_i)$
                \item Team Q-value: $Q_{tot}(s, \mathbf{a})$
                \item Monotonic mixing: $\frac{\partial Q_{tot}}{\partial Q_i} \geq 0$
                \item Ensures optimal joint action
            \end{itemize}
        \end{column}
        \begin{column}{0.5\textwidth}
            \textbf{QPLEX Extension:} \autocite{wang_qplex_2021}
            \begin{itemize}
                \item Removes monotonicity constraint
                \item Uses advantage decomposition
                \item Better handles negative interactions
                \item More flexible credit assignment
            \end{itemize}
        \end{column}
    \end{columns}
    
    \vfill
    \begin{alertblock}{Key Insight}
        Value decomposition enables scalable credit assignment while maintaining decentralized execution
    \end{alertblock}
\end{frame}

\begin{frame}{COMA: Counterfactual Multi-Agent Policy Gradients}
    \textbf{Innovation:} Actor-critic with counterfactual baselines \autocite{foerster_counterfactual_2018}
    \vfill
    \begin{block}{Counterfactual Advantage}
        $$A_i(s, \mathbf{a}) = Q(s, \mathbf{a}) - \sum_{a'_i} \pi_i(a'_i|o_i) Q(s, (\mathbf{a}_{-i}, a'_i))$$
    \end{block}
    \vfill
    \textbf{What this means:}
    \begin{itemize}
        \item Compare actual action vs. average over all possible actions
        \item Isolates agent $i$'s contribution to team performance  
        \item Efficient computation in single forward pass
        \item Addresses multi-agent credit assignment problem
    \end{itemize}
    \vfill
    \textbf{Applications:} Particularly effective in StarCraft unit micromanagement
\end{frame}

% ========== SECTION 3: EMERGENT COMPLEXITY ==========
\section{Emergent Complexity \& Competition}

\begin{frame}{Sequential Social Dilemmas}
    
    \textbf{Core Concept:} Multi-agent environments with temporal resource competition \autocite{leibo_multi-agent_2017}
    
    \begin{columns}
        \begin{column}{0.6\textwidth}
            \textbf{Key Properties:}
            \begin{itemize}
                \item \textbf{Social dilemma:} Individual vs. collective rationality
                \item \textbf{Temporal:} Actions affect future opportunities
                \item \textbf{Resource scarcity:} Limited resources create competition
                \item \textbf{Policy interdependence:} Agents' policies affect each other
            \end{itemize}
            
            \textbf{Examples:}
            \begin{itemize}
                \item Commons Harvest (resource depletion)
                \item Traffic coordination
            \end{itemize}
        \end{column}
        \begin{column}{0.4\textwidth}
            \centering
            % \includegraphics[width=\textwidth]{openai_chaser_runner.png}
            \animategraphics[controls, 
            autoplay, % Starts playing immediately
            autoresume, % Resumes after page focus returns
            autopause, % Pauses when page loses focus
            width=\linewidth]{20}{latex/imgs/oai-diagram/frame_-}{0}{168}
            \small{Red chasers vs. green runners: emergent strategies}
        \end{column}
    \end{columns}
\end{frame}

\begin{frame}{Natural Curricula \& Autocurricula}
    
    \begin{block}{Key Insight from OpenAI Research}
        "Multi-agent environments create natural curricula where difficulty scales with competitor skill"
    \end{block}
    
    \vfill
    
    \textbf{Autocurriculum Properties:}
    \begin{itemize}
        \item \textbf{No stable equilibrium:} Continuous pressure for improvement
        \item \textbf{Automatic difficulty scaling:} Environment becomes harder as agents improve  
        \item \textbf{Emergent complexity:} Simple rules → complex behaviors \autocite{bansal_emergent_2018}
        \item \textbf{Robust strategies:} Agents must generalize across opponent types
    \end{itemize}
    
    \vfill
    
    \begin{columns}
        \begin{column}{0.5\textwidth}
            \textbf{Benefits:}
            \begin{itemize}
                \item Self-generated training data
                \item Avoids over-specialization
                \item Promotes strategic diversity
            \end{itemize}
        \end{column}
        \begin{column}{0.5\textwidth}
            \textbf{Challenges:}
            \begin{itemize}
                \item Training instability
                \item Evaluation difficulties
                \item Strategy cycling
            \end{itemize}
        \end{column}
    \end{columns}
\end{frame}

% ========== SECTION 4: IMPLEMENTATION ==========
\section{Our Implementation: PPO in Simple Spread}

\begin{frame}{Experimental Setup}
    
    \begin{columns}
        \begin{column}{0.6\textwidth}
            \textbf{Environment: Simple Spread}
            \begin{itemize}
                \item 3 agents, 3 landmarks
                \item Goal: Cover all landmarks
                \item Avoid collisions
                \item Partial observability
            \end{itemize}
            
            \vfill
            \textbf{Resource Competition:}
            \begin{itemize}
                \item Spatial positions as contested resources
                \item Coordination vs. competition trade-off
                \item Credit assignment challenge
            \end{itemize}
        \end{column}
        \begin{column}{0.4\textwidth}
            \textbf{Algorithm: PPO}
            \begin{itemize}
                \item Centralized training
                \item Decentralized execution  
                \item Shared critic network
                \item Individual policy networks
            \end{itemize}
            
            \vfill
            \textbf{Why PPO?}
            \begin{itemize}
                \item Surprisingly effective in cooperative MARL \autocite{yu_surprising_2022}
                \item Stable training
                \item Minimal hyperparameter tuning
            \end{itemize}
        \end{column}
    \end{columns}
\end{frame}

\begin{frame}{Training Results: Learning Curves}
    
    \begin{center}
        \includesvg[width=0.9\textwidth]{latex/imgs/training_history.svg}
    \end{center}
    
    \textbf{Key Observations:}
    \begin{itemize}
        \item \textbf{Episode Rewards:} Convergence from -50 to +20 over 30k timesteps
        \item \textbf{Episode Length:} Decreasing length indicates faster coordination
        \item \textbf{Loss Convergence:} Both value and policy losses stabilize
    \end{itemize}
\end{frame}

\begin{frame}
    \frametitle{Training Results: Coordination Dashboard}
    
    \begin{center}
        \includesvg[width=0.9\textwidth]{latex/imgs/dashboard.svg}
    \end{center}
    
    \textbf{Coordination Development:}
    \begin{itemize}
        \item \textbf{Reward progression:} Steady improvement in team performance
        \item \textbf{Agent behaviors:} Individual policies learning complementary strategies
        \item \textbf{Coordination score:} Emergent cooperation without explicit communication
    \end{itemize}
\end{frame}

% ========== SECTION 5: INDUSTRY APPLICATIONS ==========
\section{Industry Applications \& Real-World Impact}

\begin{frame}{Cloud Computing \& Network Management}
    
    \begin{columns}
        \begin{column}{0.6\textwidth}
            \textbf{Resource Allocation Challenges:}
            \begin{itemize}
                \item CPU, memory, bandwidth allocation
                \item Load balancing across servers
                \item Network routing optimization
                \item Energy efficiency goals
            \end{itemize}
            
            \vfill
            \textbf{MARL Applications:} \autocite{zhang_multi-agent_2025}
            \begin{itemize}
                \item Autonomous resource managers
                \item Distributed decision making
                \item Real-time adaptation
                \item Scalable to large systems
            \end{itemize}
        \end{column}
        \begin{column}{0.4\textwidth}
            \textbf{Benefits:}
            \begin{itemize}
                \item Reduced latency
                \item Better resource utilization
                \item Fault tolerance
                \item Adaptive scaling
            \end{itemize}
            
            \vfill
            \textbf{Challenges:}
            \begin{itemize}
                \item High-dimensional state spaces
                \item Safety constraints
                \item Real-time requirements
                \item Partial observability
            \end{itemize}
        \end{column}
    \end{columns}
\end{frame}

\begin{frame}{Financial Markets \& Trading}
    
    \textbf{Multi-Agent Trading Systems:} \autocite{shavandi_multi-agent_2022}
    
    \vfill
    
    \begin{columns}
        \begin{column}{0.5\textwidth}
            \textbf{Applications:}
            \begin{itemize}
                \item Algorithmic trading strategies
                \item Market making optimization
                \item Portfolio management
                \item Risk assessment systems
            \end{itemize}
            
            \vfill
            \textbf{Resource Competition:}
            \begin{itemize}
                \item Limited market liquidity
                \item Information asymmetries
                \item Execution timing
                \item Capital allocation
            \end{itemize}
        \end{column}
        \begin{column}{0.5\textwidth}
            \textbf{MARL Advantages:}
            \begin{itemize}
                \item Adaptive to market regimes
                \item Strategic interaction modeling
                \item Emergent market behaviors
                \item Robust to opponent strategies
            \end{itemize}
            
            \vfill
            \textbf{Considerations:}
            \begin{itemize}
                \item Regulatory constraints
                \item Market impact
                \item Systemic risk
                \item Ethical concerns
            \end{itemize}
        \end{column}
    \end{columns}
\end{frame}

% ========== SECTION 6: CHALLENGES ==========
\section{Current Challenges \& Future Directions}

\begin{frame}{Scalability \& Robustness Challenges}
    
    \begin{block}{The Scalability Problem}
        Most MARL algorithms struggle with increasing number of agents
    \end{block}
    
    \vfill
    
    \begin{columns}
        \begin{column}{0.5\textwidth}
            \textbf{Technical Challenges:}
            \begin{itemize}
                \item Exponential action/state spaces
                \item Communication overhead  
                \item Training instability
                \item Sample efficiency \autocite{liu_scaling_2024}
            \end{itemize}
            
            \vfill
            \textbf{Current Approaches:}
            \begin{itemize}
                \item Hierarchical decomposition
                \item Graph neural networks
                \item Attention mechanisms
                \item Population-based training
            \end{itemize}
        \end{column}
        \begin{column}{0.5\textwidth}
            \textbf{Robustness Issues:}
            \begin{itemize}
                \item Overfitting to training opponents
                \item Brittleness to environment changes
                \item Strategy exploitation
                \item Evaluation difficulties
            \end{itemize}
            
            \vfill
            \textbf{Research Directions:}
            \begin{itemize}
                \item Domain randomization
                \item Meta-learning approaches
                \item Diverse opponent training
                \item Formal verification methods
            \end{itemize}
        \end{column}
    \end{columns}
\end{frame}

\begin{frame}{Theoretical Foundations \& Evaluation}
    
    \textbf{Missing Theoretical Understanding:} \autocite{fish_algorithmic_2025}
    
    \vfill
    
    \begin{columns}
        \begin{column}{0.6\textwidth}
            \textbf{Theory Gaps:}
            \begin{itemize}
                \item Convergence guarantees in multi-agent settings
                \item Sample complexity bounds
                \item Equilibrium concepts
                \item Optimality conditions
            \end{itemize}
            
            \vfill
            \textbf{Evaluation Challenges:}
            \begin{itemize}
                \item No standard benchmarks
                \item Opponent selection bias
                \item Metric interpretation
                \item Reproducibility issues
            \end{itemize}
        \end{column}
        \begin{column}{0.4\textwidth}
            \textbf{Emerging Solutions:}
            \begin{itemize}
                \item MALIB framework \autocite{zhou_malib_2023}
                \item Standardized evaluation protocols
                \item Game-theoretic analysis tools
                \item Open-source benchmarks
            \end{itemize}
            
            \vfill
            \textbf{Future Needs:}
            \begin{itemize}
                \item Unified theoretical framework
                \item Better evaluation metrics
                \item Reproducible research practices
            \end{itemize}
        \end{column}
    \end{columns}
\end{frame}

% ========== SECTION 7: KEY INSIGHTS ==========
\section{Key Insights \& Takeaways}

\begin{frame}{Literature Review Insights}
    
    \begin{block}{Fundamental Insights}
        Multi-agent environments fundamentally change the nature of learning
    \end{block}
    
    \vfill
    
    \textbf{From Literature:}
    \begin{itemize}
        \item \textbf{Natural curricula:} Competition creates automatic difficulty scaling
        \item \textbf{No stable equilibrium:} Continuous pressure for improvement
        \item \textbf{Centralized critics:} Enable stable learning in non-stationary environments
        \item \textbf{Value decomposition:} Solves credit assignment while maintaining decentralization
        \item \textbf{Emergent complexity:} Simple individual policies → complex team behaviors
    \end{itemize}
    
    \vfill
    
    \textbf{From Implementation:}
    \begin{itemize}
        \item \textbf{PPO effectiveness:} Surprisingly strong in cooperative multi-agent settings
        \item \textbf{Resource competition:} Emerges even in cooperative tasks (spatial positioning)
        \item \textbf{Coordination strategies:} Develop through individual learning processes
        \item \textbf{Training dynamics:} Convergence patterns reflect coordination development
    \end{itemize}
\end{frame}

\begin{frame}{Practical Implications}
    
    \begin{columns}
        \begin{column}{0.5\textwidth}
            \textbf{For Practitioners:}
            \begin{itemize}
                \item Start with PPO for cooperative tasks
                \item Use centralized training paradigm
                \item Monitor coordination metrics
                \item Consider value decomposition for credit assignment
            \end{itemize}
            
            \vfill
            \textbf{For Researchers:}
            \begin{itemize}
                \item Focus on scalability solutions
                \item Develop better evaluation frameworks
                \item Investigate theoretical foundations
                \item Address robustness challenges
            \end{itemize}
        \end{column}
        \begin{column}{0.5\textwidth}
            \textbf{Industry Applications:}
            \begin{itemize}
                \item Cloud resource management
                \item Financial trading systems
                \item Autonomous vehicle coordination
                \item Network optimization
            \end{itemize}
            
            \vfill
            \textbf{Research Priorities:}
            \begin{itemize}
                \item Scalable algorithms
                \item Safety guarantees
                \item Interpretable policies
                \item Real-world deployment
            \end{itemize}
        \end{column}
    \end{columns}
\end{frame}

% ========== CONCLUSION ==========
\section{Conclusion}

\begin{frame}{Conclusion: The Future of Multi-Agent RL}
    
    \begin{block}{Bottom Line}
        MARL represents a paradigm shift toward more realistic, interactive AI systems
    \end{block}
    
    \vfill
    
    \textbf{What We've Learned:}
    \begin{itemize}
        \item Multi-agent environments create unique learning challenges and opportunities
        \item Algorithmic innovations (MADDPG, QMIX, COMA) address core MARL problems
        \item Simple algorithms (PPO) can be surprisingly effective with proper implementation
        \item Real-world applications span critical infrastructure and economic systems
    \end{itemize}
    
    \vfill
    
    \textbf{The Path Forward:}
    \begin{itemize}
        \item Bridge theory-practice gap through rigorous evaluation
        \item Develop scalable solutions for large-scale systems
        \item Ensure safety and robustness for real-world deployment
        \item Foster interdisciplinary collaboration (RL + game theory + distributed systems)
    \end{itemize}
    
    \vfill
    
    \begin{center}
        \textbf{Multi-agent RL: Where competition meets cooperation, and intelligence emerges}
    \end{center}
\end{frame}

% ========== REFERENCES ==========
\section{References}

\begin{frame}[allowframebreaks]{References}
    \printbibliography[heading=none]
\end{frame}

\end{document}